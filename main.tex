\documentclass[]{article}
\usepackage {amsmath}
\usepackage[a4paper, total={7.3in, 10.3in}]{geometry}
\usepackage{cleveref}
\usepackage[dvipsnames]{xcolor}
\usepackage[titletoc]{appendix}
\usepackage{changes}
\usepackage{soul}
\usepackage{amssymb}
%\usepackage{xcolor}
\usepackage[x11names, rgb]{xcolor}
%\usepackage[utf8]{inputenc}
\usepackage{braket}
\usepackage{ulem}
\usepackage{cancel}
\usepackage{graphicx}
\usepackage{mathtools}
\usepackage{amsmath}
\usepackage{slashed}
\usepackage{amsmath}  
\usepackage{amsfonts} 
\usepackage{graphicx}
\usepackage{amssymb} 
\usepackage{amsmath}
\usepackage{mathrsfs}
\usepackage{empheq}
\usepackage{amsthm}
 \usepackage{braket}
 \usepackage{amsmath}
\DeclareMathOperator\arctanh{arctanh}
\usepackage[utf8]{inputenc}
\usepackage[english]{babel}
\usepackage{graphicx}
\newtheorem{theorem}{Theorem}[section]
\newtheorem{corollary}{Corollary}[theorem]
\newtheorem{lemma}[theorem]{Lemma}
\numberwithin{equation}{section}

\def\epp{\epsilon^{\prime}}
\def\ep{\epsilon}
\def\vep{\varepsilon}
\def\la{\langle}
\def\ra{\rangle}
\def\ppg{\pi^+\pi^-\gamma}
\def\vp{{\bf p}}
\def\ko{K^0}
\def\kb{\bar{K^0}}
\def\ka{\kappa}
\def\al{\alpha}
\def\ab{\bar{\alpha}}
\def\be{\begin{equation}}
\def\ee{\end{equation}}
\def\bea{\begin{eqnarray}}
\def\eea{\end{eqnarray}}
\def\wh{\widehat}


\title{\textbf{Yukawa theories: self energy Integral}}
\author{{Hariprashad Ravikumar}} 

\date{(For Dr.Engelhardt's meeting June 2, 2022)}
\begin{document}
	\maketitle

\section{IFD}
The Yukawa theories spin-$\frac{1}{2}$ particle self energy with scalar loop integral,
\begin{align}
    -i\Sigma(p)&=(-ie)^2\int \frac{d^dk}{(2\pi)^d}\frac{i(\slashed{k}+m)}{(k^2-m^2+i\epsilon)}\frac{-i}{(p-k)^2-\lambda^2+i\epsilon}\\
    &=-e^2\int \frac{d^dk}{(2\pi)^d}\frac{(\slashed{k}+m)}{(k^2-m^2+i\epsilon)}\frac{1}{(p-k)^2-\lambda^2+i\epsilon}
\end{align}
introducing Feynman parameter,
\begin{align}
\frac{(\slashed{k}+m)}{(k^2-m^2+i\epsilon)}\frac{1}{(p-k)^2-\lambda^2+i\epsilon}=\int_0^1dx\frac{(\slashed{k}+m)}{[k^2-2xk.p+xp^2-x\lambda^2-(1-x)m^2+i\epsilon]^2}
\end{align}
and with the change of variables $k\longrightarrow k+xp$ we find,\footnote{{https://www.physik.uzh.ch/dam/jcr:cd4d0e56-f132-4f0f-8baa-c0e863f29578/Solutions05.pdf}}
\begin{align}
    -i\Sigma(p)&=-e^2\int_0^1\int \frac{d^dk}{(2\pi)^d}\frac{\slashed{k}+x\slashed{p}+m}{[k^2-\Delta+i\epsilon]^2}
\end{align}
where, $\Delta=x\lambda^2+(1-x)m^2-x(1-x)p^2$.\\
Now the integrand in $\slashed{k}$ is odd and vanishes, while the rest can be evaluated according
to standard procedures described in Peskin $\&$ Schroeder, $\int\frac{d^dl}{(2\pi)^d}\frac{1}{(l^2-\Delta)^n}=\frac{(-1)^ni}{(4\pi)^{\frac{d}{2}}}\frac{\Gamma(n-\frac{d}{2})}{\Gamma(n)}\left(\frac{1}{\Delta}\right)^{n-\frac{d}{2}}$.
\begin{align}
    \int \frac{d^dk}{(2\pi)^d}\frac{1}{[k^2-\Delta+i\epsilon]^2}=\frac{i}{(4\pi)^\frac{d}{2}}\Gamma(2-\frac{d}{2})\Delta^{\frac{d}{2}-2}=\frac{i}{(4\pi)^{2-\zeta}}\Gamma(\zeta)\Delta^{-\zeta}
\end{align}
where, $\zeta=2-\frac{d}{2}$. Expanding at leading order around $\zeta=0$
\begin{align}
    \Gamma(\zeta)=\frac{1}{\zeta}+\mathcal{O}(1),~~x^{-\zeta}=1+\mathcal{O}(\zeta)
\end{align}
\begin{align}
    -i\Sigma(p)&=\frac{-ie^2}{(4\pi)^2}\frac{1}{\zeta}\int_0^1dx(x\slashed{p}-m)+\mathcal{O}(1)=\frac{-ie^2}{(4\pi)^2}\frac{1}{\zeta}(\frac{\slashed{p}}{2}-m)+\mathcal{O}(1)
\end{align}
In ($1+1$), $\zeta=2-\frac{2}{2}=1$
\begin{align}
    -i\Sigma(p)&=(-ie)^2\int \frac{d^2k}{(2\pi)^2}\frac{i(\slashed{k}+m)}{(k^2-m^2+i\epsilon)}\frac{-i}{(p-k)^2-\lambda^2+i\epsilon}\\
    \Aboxed{-i\Sigma(p)&=\frac{-ie^2}{(4\pi)^2}(\frac{\slashed{p}}{2}-m)+\mathcal{O}(1)}
\end{align}

\pagebreak
\section{F.Lenz's $\epsilon$-coordinates (LFD)}
On finite interval of momentum [$k^+(n)=\frac{2\pi}{L}n$], the integral will be,
\begin{align}
    -i\Sigma(p)=&-e^2\frac{1}{L}\sum_{k^+}\int_{-\infty}^{+\infty}\frac{dk^-}{(2\pi)}\frac{(\slashed{k}+m)}{(\frac{2\epsilon}{L}(k^{-})^2+2k^-k^+-m^2+i0)}\frac{1}{\frac{2\epsilon}{L}(p^--k^-)^2+2(p^+-k^+)(p^--k^-)-\lambda^2+i0}\\
    =&\lim\limits_{|\Lambda| \to \infty}\frac{-e^2}{L}\sum_{k^+}\int_{-\Lambda}^{+\Lambda}\frac{dk^-}{(2\pi)}\frac{(\slashed{k}+m)}{\left(\frac{2\epsilon}{L}\left(k^-+\frac{Lk^+}{2\epsilon}\right)^2-\frac{L(k^{+})^2}{2\epsilon}-m^2+i0\right)}\frac{1}{\left(\frac{2\epsilon}{L}\left((p^--k^-)+\frac{L(p^+-k^+)}{2\epsilon}\right)^2-\frac{L((p^+-k^{+})^2}{2\epsilon}-\lambda^2+i0\right)}\nonumber
\end{align}
let, $\tilde{k}^-=\sqrt{\frac{2\epsilon}{L}}\left(k^-+\frac{Lk^+}{2\epsilon}\right)~\Longrightarrow~dk^-=\sqrt{\frac{L}{2\epsilon}}d\tilde{k}^-$ and also  $\tilde{p}^-=\sqrt{\frac{2\epsilon}{L}}\left(p^-+\frac{Lp^+}{2\epsilon}\right)$ then
\begin{align}
    -i\Sigma(p)=&\lim\limits_{|\Lambda| \to \infty}\frac{-e^2}{L}\sqrt{\frac{L}{2\epsilon}}\sum_{k^+}\int_{\sqrt{\frac{2\epsilon}{L}}(-\Lambda+\frac{Lk^+}{2\epsilon})}^{\sqrt{\frac{2\epsilon}{L}}(+\Lambda+\frac{Lk^+}{2\epsilon})}\frac{d\tilde{k}^-}{(2\pi)}\frac{(\slashed{k}+m)}{\left((\tilde{k}^-)^2-\left(\frac{L(k^{+})^2}{2\epsilon}+m^2\right)+i0\right)}\frac{1}{\left((\tilde{p}^--\tilde{k}^-)^2-\left(\frac{L(\tilde{p}^+-k^{+})^2}{2\epsilon}+\lambda^2\right)+i0\right)}
\end{align}
let, $\left(\frac{L(k^{+})^2}{2\epsilon}+m^2\right)=\tilde{m}^2$ and $\left(\frac{L(\tilde{p}^+-k^{+})^2}{2\epsilon}+\lambda^2\right)=\tilde{\lambda}^2$. 

\begin{align}
    -i\Sigma(p)=&\lim\limits_{|\Lambda| \to \infty}-e^2\frac{1}{L}\sqrt{\frac{L}{2\epsilon}}\sum_{k^+}\int_{\sqrt{\frac{2\epsilon}{L}}(-\Lambda+\frac{Lk^+}{2\epsilon})}^{\sqrt{\frac{2\epsilon}{L}}(+\Lambda+\frac{Lk^+}{2\epsilon})}\frac{d\tilde{k}^-}{(2\pi)}\frac{(\slashed{k}+m)}{\left((\tilde{k}^-)^2-\tilde{m}^2+i0\right)}\frac{1}{\left((\tilde{p}^--\tilde{k}^-)^2-\tilde{\lambda}^2+i0\right)}
\end{align}
Introducing the Feynman parameter,
\begin{align}
\frac{(\slashed{k}+m)}{\left((\tilde{k}^-)^2-\tilde{m}^2+i0\right)}\frac{1}{\left((\tilde{p}^--\tilde{k}^-)^2-\tilde{\lambda}^2+i0\right)}&=\int_0^1dx\frac{(\slashed{k}+m)}{[(\tilde{k}^-)^2-2x\tilde{k}^-.\tilde{p}^-+x(\tilde{p}^-)^2-x\tilde{\lambda}^2-(1-x)\tilde{m}^2+i0]^2}\\
&=\int_0^1dx\frac{(\slashed{k}+m)}{[(\tilde{k}^-)^2-\tilde{\Delta}+i\epsilon]^2}
\end{align}
where, $\tilde{\Delta}=x\tilde{\lambda}^2+(1-x)\tilde{m}^2-x(1-x)(\tilde{p}^-)^2$.\\
With the change of variables $k\longrightarrow k+xp$ we find,
\begin{align}
    -i\Sigma(p)&=\lim\limits_{|\Lambda| \to \infty}\frac{-e^2}{\sqrt{2\epsilon L}}\sum_{k^+}\int_0^1 dx\int_{\sqrt{\frac{2\epsilon}{L}}(-\Lambda+\frac{Lk^+}{2\epsilon})}^{\sqrt{\frac{2\epsilon}{L}}(+\Lambda+\frac{Lk^+}{2\epsilon})}\frac{d\tilde{k}^-}{(2\pi)}\frac{\slashed{k}+x\slashed{p}+m}{[(\tilde{k}^-)^2-\tilde{\Delta}+i\epsilon]^2}
\end{align}
Now the integrand in $\slashed{k}$ is odd and vanishes, while the rest can be evaluated as
\begin{align}
     \int_{\sqrt{\frac{2\epsilon}{L}}(-\Lambda+\frac{Lk^+}{2\epsilon})}^{\sqrt{\frac{2\epsilon}{L}}(+\Lambda+\frac{Lk^+}{2\epsilon})}\frac{d\tilde{k}^-}{(2\pi)}\frac{1}{[(\tilde{k}^-)^2-\tilde{\Delta}+i\epsilon]^2}=
\end{align}



\end{document}